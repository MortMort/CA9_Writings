\subsection{Excluded component models} \label{sec:app_excl_comp_models}
This appendix contains component models which contain dynamics that from an understanding-the-theory POV are interesting but otherwise deemed unnecessary for the control objective. This is simply due to the frequency separation between the generally higher frequencies of these dynamics and the very slow eigenfrequency of a floating structure.

\subsubsection{Drivetrain (flexible)} \label{sec:mod_drt_flex}
As mentioned in \cref{sec:intro_wtcomponents} the drivetrain connects the rotor with the generator through a gearbox. The flexible drivetrain is modelled with a dampener and a spring between the rotor and the gearbox. The model is then reduced by translating the dampener and spring from the generator side to the rotor side of the gearbox. As such the model ends up consisting of two inertias coupled through a spring with stiffness $ K $ and a dampener with dampening $ B $. Due to the introduced spring and damper it is necessary to model both the rotor and and generator angle.
\begin{align} 
	J_{g} \ddot{\theta}_{gL} & = B (\dot{\theta}_r - \dot{\theta}_{gL}) + K(\theta_r - \theta_{gL}) - T_{g} \label{eq:comp_comp_drivetrain_flex_1} \\
	J_{r} \ddot{\theta}_r & = B (\dot{\theta}_{gL} - \dot{\theta}_r ) - K(\theta_r - \theta_{gL}) + T_{r} \label{eq:comp_comp_drivetrain_flex_2}
\end{align}
Recall the relationship between derivatives:
\begin{align}
	\dot{\Omega} & = \ddot{\theta}_r \\
	\dot{\omega}_{L} & = \ddot{\theta}_{gL} \\
	\dot{\theta}_g & = \omega \\
	\dot{\theta}_r & = \Omega
\end{align}
%and
%\begin{align}
%	\dot{\theta}_g & = \omega \\
%	\dot{\theta}_r & = \Omega
%\end{align}
and
\begin{align}
	\dot{\theta}_g 	&= \left(\dfrac{N_g}{N_r}\right) \dot{\theta}_{gL} \label{eq:comp_comp_drivetrain_flex_mod_3} \\
	J_{gL} 			&= J_{g} \left(\dfrac{N_r}{N_g}\right)^2 \label{eq:comp_comp_inertiamap_flex}
\end{align}
Which leaves the flexible drivetrain model as:
%\begin{align} 
%	\dot{\omega}_{gL} & = \dfrac{-B \dot{\theta}_{gL} + K(\theta_r - \theta_{gL}) - T_{g}}{J_{g}} \label{eq:comp_comp_drivetrain_flex_mod_1} \\
%	\dot{\Omega} & = \dfrac{-B \dot{\theta}_{gL} -B - K(\theta_r - \theta_{gL}) + T_{r}}{J_{r}} \label{eq:comp_comp_drivetrain_flex_mod_2} \\
%\end{align}
\begin{align} 
	\dot{\omega} & = \dfrac{B \left(\Omega - \dfrac{N_r}{N_g}\omega\right) + K\left(\theta_r - \dfrac{N_r}{N_g} \theta_{g}\right) - T_{g}}{\left(\dfrac{N_r}{N_g}\right)^2 J_{g} \dfrac{N_r}{N_g} } \label{eq:comp_comp_drivetrain_flex_mod_1} \\
	\dot{\Omega} & = \dfrac{B \left(\dfrac{N_r}{N_g}\omega - \Omega \right) \omega + K\left(\dfrac{N_r}{N_g} \theta_{g} - \theta_r\right) + T_{r}}{J_{r}} \label{eq:comp_comp_drivetrain_flex_mod_2} \\
		\dot{\theta}_g & = \omega \\
		\dot{\theta}_r & = \Omega
\end{align}
The model is observed to be linear and thus no linearisation is necessary.

The component inputs are $ \{T_g, T_r\} $ and the outputs are $ \{\omega, \Omega\} $. 

\subsubsection{Pitch system dynamics} \label{sec:comp_pitch_dyn}
The pitch system dynamics can be approximated with a simple first order low-pass filter which in the frequency domain is:
\begin{equation}\label{eq:comp_pitch_freq_dyn}
	\theta(s) = \dfrac{1}{\tau_{pit} s + 1} (\theta_{ref}(s) + \theta_{fatd}(s))
\end{equation}
where $ \tau_{pit} $ is a time-constant which suits the response of the pitching system at the operating point. In VTS the pitch system dynamics are modelled from a table which takes in the flap-wise bending moment of the blade and a pitch angle and outputs a pitch angle rate of change. As such wtLin uses the operating point and a user defined flap-wise bending moment to calculate $ \tau_{pit} $.

In the time domain the model becomes:
\begin{equation}\label{eq:comp_pitch_time}
	\dot{\theta} =\dfrac{(\theta_{ref}(s) + \theta_{fatd}(s)) - \theta}{\tau_{pit}}
\end{equation}

% The calculation of the time-domain version:
%\begin{align}\label{eq:comp_pitch}
%	\theta(s) & = \dfrac{1}{\tau_{pit} s + 1} (\theta_{ref}(s) + \theta_{fatd}(s)) \\
%	(\tau_{pit} s + 1) \theta(s) & = (\theta_{ref}(s) + \theta_{fatd}(s)) \\
%	\tau_{pit}\theta s + \theta(s)  & = (\theta_{ref}(s) + \theta_{fatd}(s)) \\
%	\tau_{pit}\dot{\theta} + \theta  & = (\theta_{ref}(s) + \theta_{fatd}(s)) \\
%	\dot{\theta} & =\dfrac{(\theta_{ref}(s) + \theta_{fatd}(s)) - \theta}{\tau_{pit}}
%\end{align}

The component inputs are $ \{\theta_{ref}, \theta_{fatd}  \} $ and the output is $ \{\theta \} $


\subsubsection{Generator model} \label{sec:comp_generator_eff}
The generator is mechanically connected to the drivetrain and is electrically connected to the converter. It is used to control the rotor speed during PLC by means of the generator torque.

A more detailed generator model could include efficiencies. In Vestas' turbine simulator (VTS) the generator efficiencies are defined in tables and are dependent on grid output power $ P $ and generator speed $ \omega $. The output is three respective output efficiencies: 
\begin{enumerate}
	\item Mechanic efficiency: $ \eta_m(P,\omega) $
	\item Electric efficiency: $ \eta_e(P,\omega) $
	\item Auxiliary efficiency: $ \eta_a(P,\omega) $
\end{enumerate}
Where 
\begin{equation}\label{eq:comp_gen_effi_eff}
	\eta(P,\omega) = \eta_m(P,\omega) + \eta_e(P,\omega) + \eta_a(P,\omega)
\end{equation}
From the total efficiency the output grid power is:
\begin{equation}\label{eq:comp_gen_elec_pow_eff}
	P_{gen} \eta(P,\omega) = P
\end{equation}
where $ P_{gen} $ is the electrical power output of the generator.

This leaves the power loss from generator to grid to be defined as:
\begin{equation} \label{eq:comp_gen_pow_loss_eff}
	P_{loss}(P, \omega) = P_{gen} - P = \dfrac{P}{\eta(P, \omega)} - P
\end{equation}
The power of a rotating machine can be defined as the product of torque and rotational velocity:
\begin{equation}\label{eq:comp_power_in_rot_eff}
	P_{gen} = T_g \omega
\end{equation}
As such for the system at hand the torque can be defined by rearranging \cref{eq:comp_power_in_rot_eff} and substituting in $ P_{gen} $ from \cref{eq:comp_gen_pow_loss_eff}. This leaves the non-linear generator model to be:
\begin{equation}\label{eq:comp_gen_torque_eff}
	T_g(P, \omega) = \dfrac{P_{loss}(P, \omega) + P}{\omega}
\end{equation}
\cref{eq:comp_power_in_rot_eff} is the non-linear model of the generator. It contains $ P_{loss}(P,\omega) $ which from \cref{eq:comp_gen_pow_loss_eff} is dependent on $ \eta(P, \omega) $. The $ \eta $ function is extracted from VTS and linearized.

The linear model of the generator is obtained through a taylor expansion. The notation is relaxed a bit such that $ P_{loss}( P, \omega) $ is simply expressed as $ P_{loss} $.
\begin{equation}\label{eq:comp_taylor_eff}
	T_g( P, \omega) \approx T_g(P_o, \omega_o) + 
	\left. \dfrac{\partial T_g( P, \omega)}{\partial P} \right |_{P_o,\omega_o} ( P-P_o) + 
	\left. \dfrac{\partial T_g( P, \omega)}{\partial \omega} \right |_{P_o,\omega_o} (\omega - \omega_o)
\end{equation}
Below the the generator torque sensitivity to the grid power change term from \cref{eq:comp_taylor_eff} is derived. From \cref{eq:comp_gen_1_1_eff} to \cref{eq:comp_gen_1_2_eff} the \textit{sum rule} is used to split the derivative into two added derivatives. From \cref{eq:comp_gen_1_2_eff} to \cref{eq:comp_gen_1_3_eff} the first fractions in the denominators of the partial derivatives of the two terms are treated as a product of two functions thus the \textit{product rule} is used. The assumption is that the grid power is completely disconnected from the generator through the converter. Thus from \cref{eq:comp_gen_1_3_eff} to \cref{eq:comp_gen_1_4_eff} $ \, \frac{\partial \, \omega^{-1}}{\partial P} = 0 $.
\begin{align} 
	\dfrac{\partial T_g( P, \omega)}{\partial P} &= \dfrac{\partial \left (\dfrac{P_{loss} +  P}{\omega}\right )}{\partial P} \label{eq:comp_gen_1_1_eff} \\
	& = \dfrac{\partial \left (\dfrac{P_{loss}}{\omega} \right )}{\partial P} + \dfrac{\partial \left ( \dfrac{ P}{\omega} \right )}{\partial P} \label{eq:comp_gen_1_2_eff} \\
	& = \dfrac{1}{\omega} \cdot \dfrac{\partial P_{loss}}{\partial P} + \dfrac{\partial \left ( \dfrac{1}{\omega} \right )}{\partial P} P_{loss} + \dfrac{1}{\omega} \cdot \dfrac{\partial P}{\partial P} + \dfrac{\partial \left (\dfrac{1}{\omega} \right )}{\partial P}  P \label{eq:comp_gen_1_3_eff} \\
	& = \dfrac{1}{\omega} \cdot \dfrac{\partial P_{loss}}{\partial P} + \dfrac{1}{\omega} \label{eq:comp_gen_1_4_eff}
\end{align}


The generator torque sensitivity to rotational velocity change from \cref{eq:comp_taylor_eff} is then derived:
\begin{align}
	\dfrac{\partial T_g(P, \omega)}{\partial \omega} & = \dfrac{\partial \left (\dfrac{P_{loss} +  P}{\omega}\right )}{\partial \omega} \\
	& = \dfrac{\partial \left (\dfrac{P_{loss}}{\omega} \right )}{\partial \omega} + \dfrac{\partial \left (\dfrac{P}{\omega} \right )}{\partial \omega} \\
	& = \dfrac{1}{\omega} \cdot \dfrac{\partial P_{loss}}{\partial \omega} + \dfrac{\partial \left (\dfrac{1}{\omega} \right)}{\partial \omega} P_{loss} + \dfrac{1}{\omega} \dfrac{\partial P}{\partial \omega} + \dfrac{\partial \left (\dfrac{1}{\omega} \right )}{\partial \omega} P \\
	& = \dfrac{1}{\omega} \cdot  \dfrac{\partial P_{loss}}{\partial \omega} - \dfrac{1}{\omega^2}(P + P_{loss}) + \dfrac{1}{\omega} \dfrac{\partial P}{\partial \omega} \\
	& = -\dfrac{1}{\omega^2}(P + P_{loss}) + \dfrac{1}{\omega} \cdot \dfrac{\partial P_{loss}}{\partial \omega}
\end{align}
The above derived generator model is referred to the low-speed generator side by replacing $ \omega $ with $ \omega_L $ where $ \omega_L = \left (\frac{N_r}{N_g} \right ) \omega $. This yields the final linear generator model evaluated at the operating point $ (P_o, \omega_o) $. Furthermore $P_{loss}$ is still just a function of $ \omega_L $ and $ P $:
\begin{equation}
	\begin{split}
		T_g(P, \omega_L) 	& \approx \left. \left ( \dfrac{1}{\omega_L} \cdot \dfrac{\partial P_{loss}(P, \omega_L)}{\partial P} + \dfrac{1}{\omega_L} \right ) \right |_{P_o,\omega_{L_o}} (P - P_0) \\ 
		& + \left ( -\dfrac{1}{\omega_L^2}(P + P_{loss}(P, \omega_L)) + \left. \dfrac{1}{\omega_L} \cdot \dfrac{\partial P_{loss}(P, \omega_L)}{\partial \omega_L} \right ) \right |_{P_o,\omega_{L_o}} (\omega_L - \omega_{L_o})
	\end{split}
\end{equation}
The calculation of the power loss and derivation of the power loss sensitivity to grid power are left out here, but are calculated in wtLin at an operating point. In the wtLin tool these sensitivities are calculated based on the extracted tables of efficiency $ \eta $.

Presently the generator model in wtLin is implemented assuming a stiff drivetrain such that $ \Omega = \omega_L $.

The component model input is $ \{P, \Omega\} $ and the output is $ \{T_g\} $



\subsection{Fore-aft tower model fitting} \label{sec:app_mod_foreaft_fitting}
As described in \hyperref[sec:comp_foreaft_mod]{\textbf{fore-aft tower model}} \cref{sec:comp_foreaft_mod} the component which models the fore-aft movement consists of a simple second order mass-spring-damper system. This is obviously a seriously simplified model which consists of only three parameters: A mass $ m $, a spring constant $ k $ and a damper constant $ b $. Setting these parameters such that the model fits the behaviour of the real system well is not intuitive. When the model equations were written on the standard second order TF form it became apparent that the parameters could be derived from the mass, the natural frequency $ \omega_n $ and the dampening factor $ \zeta $. While these parameters are much more intuitive to place some tuning still has to be done. This section is dedicated to explaining the tuning procedure and showcasing the relevant tuning results.

\medskip

\noindent Setting the natural frequency around the eigenfrequency of the turbine is a good place to start.

\smallskip
\noindent A first guess for the effective mass $ m $ is the combined mass of: The tower, nacelle, hub and rotor blades. This is simply a rough estimate which is not expected to yield satisfactory results mainly because the mass M does not represent the actual mass of the system but rather the inertia of the pitching of the whole structure in the water. 

\smallskip
\noindent Many factors affect the dampening of the fore-aft movement. As described in \cref{sec:intro_theFOWT} ballast, buoyancy and mooring line forces all contribute to the stability and dampening of the fore-aft tower movement. Furthermore as also described in \cref{sec:theory_fowt_challenges} the rotor blades act as a sail which dampens the movement in the surge direction. A low dampening factor $ \zeta $ is assumed as a start since the dampening from the blades is modelled in the \hyperref[sec:comp_aero_thrust]{\textbf{aerodynamic thrust model}} \cref{sec:comp_aero_thrust}. Thus the only contributors to dampening of the system are from the other mentioned forces.

\medskip
In order to be able to fit the model to the real system it is of course necessary to have some data to fit it against. This is where \textit{SysIdFreqSweep} enters the picture.

\subsubsection{System identification}
A Vestas tool called \textit{SysIdFreqSweep} is utilized to get frequency response plots of the real system from specific actuator inputs to any sensor in the simulation environment. The tool functionality can be split into to main parts. 

The \textbf{first part} is the generation of setup files which alter the simulation environment such that a sinusoid of chosen frequency and amplitude is induced on a chosen reference. A full simulation is run for every frequency between a start and end frequency. A preliminary test was made where the system behaviour was analyzed when frequencies were induced, in the appendix in \cref{app:tj_00}. In this test only a few frequencies were induced on the rotor speed reference and the effect on the rotor speed and fore-aft motion was commented.

The \textbf{second part} is the processing of the simulation output data to plot frequency responses from the actuator to an output sensor. 

\medskip
For the parameter tuning the input actuators were chosen to be the generator speed reference and the pitch angle reference and the observed outputs were the measured generator speed and the tower top velocity.

60 different frequencies were spaced logaritmacally between 0.01 Hz and 0.3 Hz. The amplitude of the sinusoid was set to 5\% of the mean generator speed reference and for the rotor pitch it was set to 0.5 degrees \todo[inline]{Virker ikke p.t. Vi må se om det kommer til at virke.}
