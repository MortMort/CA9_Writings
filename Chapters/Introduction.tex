\section{Introduction} \label{sec:intro}
In this chapter a general introduction of the the xxxxx

\subsection{State of affairs and motivation}
The global average temperature of the earth is rising mainly attributed to the greenhouse effect caused primarily by carbon dioxide (CO2) emissions. The consequences of these temperature changes are aplenty and the more obvious signs of change are already starting to show. Record high temperatures, droughts, storms, floods and other extreme weather conditions and events are becoming more and more frequent. The main contributor of human CO2 emissions is the energy section which represents 76\% of emissions with 42\% resulting from electric energy production \cite{wri2018}. The energy demand on a global scale is still steadily increasing. Due to COVID-19 an unprecedented drop in energy use and CO2 emissions were observed in 2020. But primary energy demand increased again in 2021 by 5.8 \% which was 1.3 \% higher than in 2019 \cite{bp2022}. Fossil fuel still makes up 82 \% of the primary energy use in 2021 \cite{bp2022}. A transition from fossil energies to renewable energy sources is one of the most efficient remedies to lower CO2 emissions and solar and wind energy are widely accepted as some of the best green energy alternatives. Wind and solar reached a 10.2\% share of power generation in 2021 \cite{bp2022}. As of 2021 236 GW of wind power capacity is installed in Europe with 12\% being offshore. 17 GW was installed in 2021 alone with a 19 \% share being offshore wind\cite{Sesto1992}. \\

Despite the higher levelized cost of energy (LCOE) og offshore wind turbines (WTs) compared to the onshore counterpart the trend towards offshore wind is increasing. There are sensible reasons for this. Offshore wind is on average 20\% faster than onshore. Turbulence is also less due to the lack of obstacles at sea which could potentially extend the expected WT lifetime \cite{Christiansen2013}. Furthermore wind farms at sea do not have the same clearance issues with regards to minimum distance from urban areas and houses. As a result visual and noise annoyances are also decreased. Issues with regards to offshore WTs is the shallow water debt requirement of most types of foundations. Above 50 meters water debt fixed-bottom offshore WTs start to become economically infeasible \cite{Lefebvre2012}. Shallow water debt sites will also eventually exhaust and it will become a necessity to install WTs at deeper waters. This is where the floating offshore wind turbine (FOWT) comes into play. Floating turbines have mainly been a research subject in the prototype stage since the first Blue H Technologies FOWT was started up in 2007. It was placed 21.3 km from the coast of Apulia, Italy and was a prototype installed in 113 meters waters to gather test data on wind and sea conditions. It was decommissioned at the end of 2008. Despite not having reached full commercial feasibility yet, the FOWT park scale is increasing. Kincardine, the largest floating WT project, was commissioned in 2021 off the coast of Scotland in 60-80 meter waters. The project features 5 Vestas V164-9.5 MW and one V80-2 MW turbine with a combined nominal output of 48 megawatts (MW). 

\subsection{Wind turbine components}
In \cref{fig:wt_components} a digram of the main components of a WT is seen. Since this report is focused on rotor speed control the most relevant components are the: Blades and rotor, pitch system and generator. Almost all other components also play a role and most will more or less be mentioned throughout the report. Both onshore and fixed-bottom offshore turbines are connected to a firm foundation. Therefore the primary external loads are the wind and gravity. 

When the wind, which is also pictured on the diagram, strikes the rotor blades a torque is generated which drives the turbine rotor around. The rotor spins the low-speed shaft which in turn spins the generator through the high-speed shaft. Depending on whether the turbine is operating below or above rated wind speed it will regulate the rotor rotational velocity with either the generator torque or the blade pitch angle. Control at below rated wind speed called partial load control (PLC) is done through regulating the generator torque. The blade pitch is sat at an optimal angle. Control at above rated wind speed called full load control (FLC) is done by regulating the blade pitch angle.
\begin{figure}[h]
	\centering
	\includegraphics[width=0.7\linewidth]{Graphics/WtComponents.png}
	\caption{Illustration with an overview of the main components which make up a wind turbine}
	\label{fig:wt_components}
\end{figure}

\subsection{The floating offshore wind turbine}
Floating turbines are characterized by having a floating foundation in contrast to the fixed-bottom foundation. Many foundation concepts exist and new ones are continuously proposed but four types are mainly used currently and are depicted in \cref{fig:floating_concepts}. Each floater type is categorized based on which force is the main driver in keeping the turbine upright in the face of external forces.
\begin{figure}[h]
	\centering
	\includegraphics[width=1\linewidth]{Graphics/FloatingFoundationConcepts.jpg}
	\caption{The four main floating foundation concepts: Semi-submersible, Spar boyo, Tension leg platform and the Barge \cite{DNV-GL2018}}
	\label{fig:floating_concepts}
\end{figure}

A characteristic property of floating platforms is static stability which ensures that the overturning moment from the external loads is counteracted. The restoring force can be expressed as a sum of three forces:
\begin{equation} \label{eq:F_rest}
	F_{rest} = F_{WP} + F_B + F_{moor}
\end{equation}
where \smallskip 
\begin{center}
	\begin{tabular}{l p{12cm}}
		$ F_{WP} $ & is the waterplane contribution which is dependent on the area of water occupied by the floater \\
		$ F_{B} $ & is the ballast contribution which is dependent on the distance from the center of gravity and the center of buoyancy of the platform.\\
		$ F_{moor} $ & is the mooring line system contribution.
	\end{tabular}
\end{center} \smallskip
Foundation types in which the waterplane contribution prevails are buoyancy-stabilized. Likewise foundations where the ballast and mooring line contributions prevail are ballast and mooring line stabilized respectively. The semi-submersible is both buoyancy and ballast stabilized due to its platform sitting fairly deep XXXXX
Floaters are subjected to additional external loads: Hydrodynamic forces act on the foundation and are comprised of contributions from waves, buoyancy forces and viscous forces. The mooring system will furthermore affect the foundation and differently depending on the type of mooring system.\\

FOWTs are still in the developmental stage with a higher LCOE than the on- and offshore counterparts. They experiences greater structural strain because of significant rotational and translational motions of the support structure. The illustration in \cref{fig:fowt_coordinates} shows the degrees of freedom (DOFs) of a FOWT. 
\begin{figure}[h]
	\centering
	\includegraphics[width=0.55\linewidth]{Graphics/FOWTcoordinates.png}
	\caption{The 6 degrees of freedom (DOFs) of a floating offshore wind turbine. (This is not the specific WT which is modelled in this report) \cite{Vanelli2021}}
	\label{fig:fowt_coordinates}
\end{figure}
Because of the increased movement, the tower is reinforced to handle the greater loads. This drives the LCOE up and as such it is of great interest to explore methods which could be used to dampen the movement of the floating WT structure. One of the large challenges in the FOWT development is handling the negative dampening problem: As the wind is flowing through the rotor plane the thrust of the wind shoves and pitches the WT foundation backwards. The backwards thrust causes the foundation pitch to oscillate thus changing the relative velocity as seen by the rotor. When the turbine moves forward the relative wind speed is higher resulting in a greater torque and thrust. Above nominal wind speeds the rotor speed is regulated via the rotor pitch angle. Thus the rotor speed controller will pitch the blades out of the wind to decrease the rotational speed but at the same time also decreases the thrust. Thus the WT thrusts forwards even faster as a result of the reduced dampening resulting from the higher pitch angle. As the turbine reaches the peak of its forward motion and begins to move backwards the opposite occurs. The 



\subsection{Problem definition}
Can a state-space controller, applied to a linear control model be used to reduce the fore-aft motion of a floating offshore wind turbine? (Subject to change!)


