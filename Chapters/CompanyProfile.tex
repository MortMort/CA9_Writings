\section{Company Profile and reflection} \label{company}
As part of AAUs project oriented stay at a company (POSC) the student is expected to deliver a company and project process related description as part of the project documentation. This chapter firstly includes a general description of Vestas as a company including its areas of work, organisational layout and culture. Secondly a description of the student's personal payoff from the work at Vestas is included, divided into three parts: Theoretical and practical, work-related and social. Finally suggestions to changes to POSC procedures are listed followed by a reflection on sharing of knowledge between Vestas and AAU.


\subsection{Description of Vestas}

- What is Vestas' product / work areas
- What is the organisational layout of Vestas
- What is Vestas' work culture



\subsection{Personal payoff from stay at Vestas}
My payoff from my project oriented stay at Vestas split into three main parts: The theoretical/practical payoff, the work-related payoff and the social payoff.



\subsubsection{Theoretical and practical}
Almost all of the theoretical knowledge i have gained throughout my work on the project at Vestas is documented in this report. This subsection will attempt to summarise this knowledge in a manageable form.

\smallskip
Before i started my work i had limited knowledge on wind turbines although i knew the very general layout of WT components and the different turbine configuration types. I also understood the basic WT function and operation with regards to wind conversion to electrical energy in the generator and the subsequent delivery to the grid through the converter. All of my knowledge was although surface level.

To be able to work with the floating problem i had to build a firm foundation of understanding of WTs and WT control on a structural/loads-related level. A list of theoretical gains include but are not limited to knowledge on:
\begin{itemize}
	\item How to model the processes that relate to the energy transfer from the wind to the forces and torques experienced by the turbine rotor.
	\item How a WT controller can work by controlling the turbine based on different operating regions that relate to the wind speed namely PLC and FLC.
	\item How the classical PLC and FLC controllers are set up and what defines the classical operating regions in PLC and FLC.
	\item What periodic disturbances and exciting forces to be aware of with regards to WT control and disturbance rejection.
	\item What causes the negative dampening problem present on FOWTs and how to deal with it.
	\item How to make a simple control-oriented model of a WT in an operating point that captures problem-relevant dynamics while leaving out irrelevant dynamics.
	\item How to make an LQI controller which minimizes system state deviations from an operating point (OP) which contains a reference which is constant.
	\item How the LQI controller might move the closed system poles when changing the $ Q $ and $ R $ weights in a manner which is not obvious or in line with initial hypothesis.
	\item How to use Bryson's Rule to make good initial guesses and tuning LQI $ Q $ and $ R $ matrix weights to get good performance.
	\item How to use simulation tools such as Vestas' VTS to simulate wind turbine loads to both understand turbine behaviour and to tune WT controllers.
\end{itemize}

Practical:
General limitations

- LaC tools

Generelt om vindturbiner (opbygning, funktion), kontrol af vindturbiner (kontrolområder, klassiske udfordringer), etc.))

\subsubsection{Work-related}
What is meant by work-related payoff is: What experience or insight has been gained by being at and working at Vestas in contrast to studying at and being at AAU. Doing work at a company is under normal circumstances different than at a university. There are usually expectations from co-workers or a boss which will expect an output which in some sense creates value for the company or other co-workers. Thus besides perhaps the interest in theoretical knowledge there is further motivation for having good performance. This is unlike work at a university where the sole motivation for putting in work is to gain knowledge, live up to the expectations of project team members and perhaps to pass the exams. In a "\textit{normal}" internship the intern takes on tasks which are present and relevant in the company at the time of the internship. This is different from AAUs POSC because the POSC includes a predefined project proposal which is followed throughout the whole semester. One will therefore come to the conclusion that the POSC work is much like any other project done at AAU. The only major difference is that the project work is done at a company and that you are alone about and can only discuss with co-workers and your POSC supervisor. You are not necessarily participating in the daily project matters and meetings which happen at the company because they are most likely irrelevant to your pre-defined project work.

Therefore I can conclude that on a work-related level i have almost not learned anything i did not already know. Something worthy of mention might be that i have gotten a small insight into how specifically the LaC department runs scrum as a project management tool. This is also at a very surface level, since it has never been made relevant for me.

\subsubsection{Socially}
I have most days been doing my project work at the Vestas LaC department office. This has given me the opportunity to talk with and socialize with most of the relevant co-workers i have in LaC in Denmark. I have also had the opportunity to ask for help and discuss my project-relevant problems with co-workers when they were not very busy with other matters - which they usually were. But besides the annual Christmas party and some birthday celebrations with cake here and there i have not been invited to many social activities - inside or outside of work. But i am still slightly satisfied with the social outcome of my time at Vestas. Given that my project has never been relevant to anyone but my \textit{buddy} assigned to help me at Vestas I could not expect to form deeper friendships with co-workers. This requires time and ideally that i do some work in corporation with other co-workers. This is in contrast to work done in a project group at AAU where corporation can create fast friendships if you're lucky with your project group.

\subsection{Suggestions on changes to the curriculum and procedures}

I would suggest that the POSC did not suggest that all work to be done throughout the project is predefined but also leaves room or even suggests that the intern takes on company/department relevant tasks which would engage the student more in the daily work done in the company. As it stands POSC ends up being no different than a less anxiety inducing masters degree done at a company. If the goal was to have the student be more engaged in the daily work at the company to actually get a feeling of how it is to work at a company then the POSC guidelines should attempt to ensure that the student is assigned to tasks which are carried out with co-workers. This semester we were three students doing an internship at Vestas and the story is the same for all of us. Our work is almost completely cut off from the work done by co-workers. This problem might be more specific to Vestas than it is to POSC in general but i can only speak from my own experience and the experience of my friends here at Vestas.


\subsection{Reflection on sharing of knowledge between Vestas and AAU}
Vestas will of course have gained some knowledge through the work that I have done on the project. But ss it stands there has been no knowledge sharing between Vestas and AAU through my project work except the knowledge that my AAU supervisor might have gained by supervising me - and of course perhaps the knowledge that AAU can gain from the report I am writing. Other than that i can not think of any channel which should have resulted in knowledge sharing between Vestas and AAU. The meetings which have included both Vestas and AAU have been directly aimed at helping me progress with my project work and my report. This might be what is meant by sharing of knowledge but i do now know.
