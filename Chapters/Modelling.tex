\section{Modelling} \label{sec:mod} % Incl. Subsection "Component Models"

\subsection{wtLin model}
A Matlab tool named wtLin has been developed by the LaC department in Vestas for creating linear model based on parameters extracted from Vestas Turbine Simulator (VTS). The tool specific turbine model files and an operating point to create individual components which can be connected in loops. Each component translates one or more inputs to one or more outputs. In this section the relevant component models contained in wtLin will be outlined.\\

\subsubsection{Drivetrain (stiff)}
In the simple stiff drivetrain model it is assumed that there is no dampening or spring effects in the drivetrain between the rotor and the generator. As such if some torque is applied at the rotor resulting in a change in twist angle at the rotor the resulting twist angle at the generator is instant and directly proportional to the twist angle at the generator.

The stiff drivetrain consists of two free inertias connected through a gearbox. The drivetrain is modelled from newtons second law???? as such:
\begin{equation}\label{eq:wtlin_comp_drivetrain}
	(J_{gL} + J_{r}) \ddot{\theta}_r = T_{r} + T_{g}
\end{equation}
\begin{equation}\label{eq:wtlin_comp_drivetrain}
	(J_{gL} + J_{r}) \ddot{\theta}_r = T_{r} + T_{g}
\end{equation}
The torques are related to the low-speed rotor side and thus the high-speed generator side inertias are mapped to the low speed side:
\begin{equation} \label{eq:wtlin_comp_inertiamap}
	J_{gL} = J_{gL} = J_{gH} \left(\dfrac{N_r}{N_g}\right)^2
\end{equation}
where $ N_r $ and $ N_g $ is the number of teeth on the rotor and generator side of the gearbox respectively. The rotor spins at angular velocity $ \dot{\theta}_r $.


\subsubsection{Drivetrain (flexible)}
A more accurate drivetrain model includes dampening and spring effects in the drivetrain between the rotor and the generator. The drivetrain is modelled with a dampener and a spring is between the rotor and the gearbox and between the gearbox and the generator. The model is then reduced by translating the dampener and spring from the generator side to the rotor side of the gearbox. As such the model ends up consisting of two inertias coupled through a spring with stiffness $ K $ and a dampener with dampening $ B $ where $ K $ and $ B $ are a combination of both the rotor and generator side dampener and spring coefficients. \todo[]{Har Jesper modelleret drivetrain så at der også er fjeder og dæmper mellem gearbox og generator!?}

\begin{align} 
	J_{g} \ddot{\theta}_g & = -B \dot{\theta}_{gL} + K(\theta_r - \theta_{gL}) - T_{g} \label{eq:wtlin_comp_drivetrain_flex_1} \\
	J_{r} \ddot{\theta}_r & = -B \dot{\theta}_{gL} -B - K(\theta_r - \theta_{gL}) + T_{r} \label{eq:wtlin_comp_drivetrain_flex_2} \\
	\dot{\theta}_g & = \left(\dfrac{N_g}{N_r}\right) \dot{\theta}_{gL} \label{eq:wtlin_comp_drivetrain_flex_3}
\end{align}

\clearpage \newpage
\subsubsection{Generator model}
When considering the generator of a WT the obvious point of view (POV) is to consider it as outputting power when it is rotated. As such from this POV the input would be torque and rotational velocity and the output power would be the input to the converter. But with regards to rotor speed control the generator is on the contrary viewed as a source of torque. As such the inputs end up being rotational velocity and electrical power from the converter.

In Vestas' turbine simulator (VTS) the generator efficiencies are defined in tables and are dependent on grid output power $ P_{grid} $ and generator speed $ \omega $. The output is three respective output efficiencies: 
\begin{enumerate}
	\item Mechanic efficiency: $ \eta_m(P_{grid},\omega) $
	\item Electric efficiency: $ \eta_e(P_{grid},\omega) $
	\item Auxiliary efficiency: $ \eta_a(P_{grid},\omega) $
\end{enumerate}
Where 
\begin{equation}\label{eq:wtLin_gen_effi}
	\eta(P_{grid},\omega) = \eta_m(P_{grid},\omega) + \eta_e(P_{grid},\omega) + \eta_a(P_{grid},\omega)
\end{equation}
From the total efficiency the output grid power is:
\begin{equation}\label{eq:wtLin_gen_elec_pow}
	P_{gen} \eta(P_{grid},\omega) = P_{grid}
\end{equation}
where $ P_{gen} $ is the electrical power output of the generator.

This leaves the power loss from generator to grid to be defined as:
\begin{equation} \label{eq_wtLin_gen_pow_loss}
	P_{loss}(P_{grid}, \omega) = P_{gen} - P_{grid}% = \dfrac{P_{grid}}{\eta(P_{grid}, \omega)} - P_{grid}
\end{equation}
The power of a rotating machine can be defined as the product of torque and rotational velocity:
\begin{equation}\label{eq:wtLin_power_in_rot}
	P_{gen} = T_{gen} \omega
\end{equation}
As such for the system at hand the torque can be defined by rearranging \cref{eq:wtLin_power_in_rot} and substituting in $ P_{gen} $ from \cref{eq_wtLin_gen_pow_loss}:
\begin{equation}\label{key}
	T_{gen}(P_{grid}, \omega) = \dfrac{P_{loss}(P_{grid}, \omega) + P_{grid}}{\omega}
\end{equation}
The linear model of the generator is gained through a taylor expansion. The notation is relaxed a bit such that $ P_{loss}( P_{grid}, \omega) $ is simply expressed as $ P_{loss} $.
\begin{equation}\label{eq:wtLin_taylor}
	T_{gen}( P_{grid}, \omega) \approx T_{gen}(P_o, \omega_o) + 
	\left. \dfrac{\partial T_{gen}( P_{grid}, \omega)}{\partial P_{grid}} \right |_{P_o,\omega_o} ( P_{grid}-P_o) + 
	\left. \dfrac{\partial T_{gen}( P_{grid}, \omega)}{\partial \omega} \right |_{P_o,\omega_o} (\omega - \omega_o)
\end{equation}
Below the the generator torque sensitivity to the grid power change term is derived. From \cref{eq:wtLin_gen_1_1} to \cref{eq:wtLin_gen_1_2} the \textit{sum rule} is used to split the derivative. From \cref{eq:wtLin_gen_1_2} to \cref{eq:wtLin_gen_1_3} the first fractions in the denominators of the partial derivatives of the two terms are treated as a product of two functions thus the \textit{product rule} is used. The assumption is that the grid power is completely disconnected from the generator through the converter. Thus from \cref{eq:wtLin_gen_1_3} to \cref{eq:wtLin_gen_1_4} $ \, \dfrac{\partial \, \omega^{-1}}{\partial P_{grid}} = 0 $.
\begin{align} 
	\dfrac{\partial T_{gen}( P_{grid}, \omega)}{\partial P_{grid}} &= \dfrac{\partial \left (\dfrac{P_{loss} +  P_{grid}}{\omega}\right )}{\partial P_{grid}} \label{eq:wtLin_gen_1_1} \\
	& = \dfrac{\partial \left (\dfrac{P_{loss}}{\omega} \right )}{\partial P_{grid}} + \dfrac{\partial \left ( \dfrac{ P_{grid}}{\omega} \right )}{\partial P_{grid}} \label{eq:wtLin_gen_1_2} \\
	& = \dfrac{1}{\omega} \cdot \dfrac{\partial P_{loss}}{\partial P_{grid}} + \dfrac{\partial \left ( \dfrac{1}{\omega} \right )}{\partial P_{grid}} P_{loss} + \dfrac{1}{\omega} \cdot \dfrac{\partial P_{grid}}{\partial P_{grid}} + \dfrac{\partial \left (\dfrac{1}{\omega} \right )}{\partial P_{grid}}  P_{grid} \label{eq:wtLin_gen_1_3} \\
	& = \dfrac{1}{\omega} \cdot \dfrac{\partial P_{loss}}{\partial P_{grid}} + \dfrac{1}{\omega} \label{eq:wtLin_gen_1_4}
\end{align}

The generator torque sensitivity to rotational velocity change is derived:
\begin{align}
	\dfrac{\partial T_{gen}(P_{grid}, \omega)}{\partial \omega} & = \dfrac{\partial \left (\dfrac{P_{loss} +  P_{grid}}{\omega}\right )}{\partial \omega} \\
	& = \dfrac{\partial \left (\dfrac{P_{loss}}{\omega} \right )}{\partial \omega} + \dfrac{\partial \left (\dfrac{P}{\omega} \right )}{\partial \omega} \\
	& = \dfrac{1}{\omega} \cdot \dfrac{\partial P_{loss}}{\partial \omega} + \dfrac{\partial \left (\dfrac{1}{\omega} \right)}{\partial \omega} P_{loss} + \dfrac{1}{\omega} \dfrac{\partial P}{\partial \omega} + \dfrac{\partial \left (\dfrac{1}{\omega} \right )}{\partial \omega} P \\
	& = \dfrac{1}{\omega} \cdot  \dfrac{\partial P_{loss}}{\partial \omega} - \dfrac{1}{\omega^2}(P + P_{loss}) + \dfrac{1}{\omega} \dfrac{\partial P}{\partial \omega} \\
	& = -\dfrac{1}{\omega^2}(P + P_{loss}) + \dfrac{1}{\omega} \cdot \dfrac{\partial P_{loss}}{\partial \omega}
\end{align}


\subsubsection{Unity model converter}
The converter is regarded as a source of electrical whose output is the power which is input to the generator.

The dynamics of modern converters are way faster than the rotor and tower dynamics and therefore it is simply modelled as a direct feed-through:
\begin{equation}\label{eq:wtLin_comp_convdft}
	P_{conv} = P_{ref}
\end{equation}
In other words the converter is treated as a \textit{black box} system which, when given a power reference, delivers a power equal to said power reference instantly.



% Template:
%\begin{equation}\label{eq:wtlin_comp_}
%	
%\end{equation}

