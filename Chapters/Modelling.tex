\section{Modelling} \label{sec:mod} % Incl. Subsection "Component Models"
The purpose of this section is to outline the model of the system. The model is intended to be used as a control model for minimizing the turbine fore-aft motion. The fore-aft motion dynamics of floating wind turbines are exceedingly slow (a period of 30 seconds is normal) and the modelling is approached with this in mind: Dynamics which are relatively fast compared to the fore-aft motion dynamics are left out since they are of little importance to the control objective. In stead such components are modelled with algebraic equations. 

Modelling is approached from a component point of view where parts of the system are modelled individually. Each component takes a set of inputs and calculates a set of outputs. If the component contain dynamics they include internal states. Non-linear components are linearised individually at an operating point. Model parameters and operating points are extracted from Vestas Turbine Simulator (VTS) through wtLin.


\subsection{wtLin model}
A Matlab tool named wtLin has been developed by the LaC department in Vestas for creating a linear model based on parameters extracted from VTS. The tool contains component models which can be connected in loops based on the input-output variables of each component model. A component can be in the form of both specific mechanical models such as a generator and other phenomena which are relevant for the system such as the interaction between the tower movement and the free wind.

The tool takes in an operating point and a specific set of components and outputs a connected linear model at that operating point.

In this section the relevant component models are derived.


\subsubsection{Drivetrain (stiff)}
As mentioned in \cref{sec:intro_wtcomponents} the drivetrain connects the rotor with the generator through a gearbox. In the simple stiff drivetrain model it is assumed that there is no dampening or spring effects in the drivetrain between the rotor and the generator. As such if some torque is applied at the rotor resulting in a change in twist angle at the rotor the resulting twist angle at the generator is instant and directly proportional to the twist angle at the generator.

The stiff drivetrain consists of two free inertias connected through a gearbox. The drivetrain is modelled from newtons second law for rotation as such:
\begin{equation}\label{eq:wtlin_comp_drivetrain}
	(J_{gL} + J_{r}) \ddot{\theta}_r = T_{r} + T_{g}
\end{equation}
The torques are related to the low-speed rotor side and thus the high-speed generator side inertias are mapped to the low speed side:
\begin{equation} \label{eq:wtlin_comp_inertiamap}
	J_{gL} = J_{gL} = J_{gH} \left(\dfrac{N_r}{N_g}\right)^2
\end{equation}
where $ N_r $ and $ N_g $ is the number of teeth on the rotor and generator side of the gearbox respectively. The rotor spins at angular velocity $ \dot{\theta}_r $.


\subsubsection{Drivetrain (flexible)}
A more accurate drivetrain model includes dampening and spring effects in the drivetrain between the rotor and the generator. The drivetrain is modelled with a dampener and a spring is between the rotor and the gearbox and between the gearbox and the generator. The model is then reduced by translating the dampener and spring from the generator side to the rotor side of the gearbox. As such the model ends up consisting of two inertias coupled through a spring with stiffness $ K $ and a dampener with dampening $ B $ where $ K $ and $ B $ are a combination of both the rotor and generator side dampener and spring coefficients. \todo[]{Har Jesper modelleret drivetrain så at der også er fjeder og dæmper mellem gearbox og generator!?}

\begin{align} 
	J_{g} \ddot{\theta}_g & = -B \dot{\theta}_{gL} + K(\theta_r - \theta_{gL}) - T_{g} \label{eq:wtlin_comp_drivetrain_flex_1} \\
	J_{r} \ddot{\theta}_r & = -B \dot{\theta}_{gL} -B - K(\theta_r - \theta_{gL}) + T_{r} \label{eq:wtlin_comp_drivetrain_flex_2} \\
	\dot{\theta}_g & = \left(\dfrac{N_g}{N_r}\right) \dot{\theta}_{gL} \label{eq:wtlin_comp_drivetrain_flex_3}
\end{align}

\clearpage \newpage
\subsubsection{Generator model}
When considering the generator of a WT the obvious point of view (POV) is to consider it as outputting power when it is rotated. As such from this POV the input would be torque and rotational velocity and the output power would be the input to the converter. But with regards to rotor speed control the generator is on the contrary viewed as a source of torque. As such the inputs end up being rotational velocity and electrical power from the converter.

In Vestas' turbine simulator (VTS) the generator efficiencies are defined in tables and are dependent on grid output power $ P $ and generator speed $ \omega $. The output is three respective output efficiencies: 
\begin{enumerate}
	\item Mechanic efficiency: $ \eta_m(P,\omega) $
	\item Electric efficiency: $ \eta_e(P,\omega) $
	\item Auxiliary efficiency: $ \eta_a(P,\omega) $
\end{enumerate}
Where 
\begin{equation}\label{eq:wtLin_gen_effi}
	\eta(P,\omega) = \eta_m(P,\omega) + \eta_e(P,\omega) + \eta_a(P,\omega)
\end{equation}
From the total efficiency the output grid power is:
\begin{equation}\label{eq:wtLin_gen_elec_pow}
	P_{gen} \eta(P,\omega) = P
\end{equation}
where $ P_{gen} $ is the electrical power output of the generator.

This leaves the power loss from generator to grid to be defined as:
\begin{equation} \label{eq_wtLin_gen_pow_loss}
	P_{loss}(P, \omega) = P_{gen} - P% = \dfrac{P}{\eta(P, \omega)} - P
\end{equation}
The power of a rotating machine can be defined as the product of torque and rotational velocity:
\begin{equation}\label{eq:wtLin_power_in_rot}
	P_{gen} = T_{gen} \omega
\end{equation}
As such for the system at hand the torque can be defined by rearranging \cref{eq:wtLin_power_in_rot} and substituting in $ P_{gen} $ from \cref{eq_wtLin_gen_pow_loss}:
\begin{equation}\label{key}
	T_{gen}(P, \omega) = \dfrac{P_{loss}(P, \omega) + P}{\omega}
\end{equation}
The linear model of the generator is gained through a taylor expansion. The notation is relaxed a bit such that $ P_{loss}( P, \omega) $ is simply expressed as $ P_{loss} $. It is assumed that the system is linearised around an operating point and thus the $ T_{gen}(P_o, \omega_o) $ part is equal to zero.
\begin{equation}\label{eq:wtLin_taylor}
	T_{gen}( P, \omega) \approx T_{gen}(P_o, \omega_o) + 
	\left. \dfrac{\partial T_{gen}( P, \omega)}{\partial P} \right |_{P_o,\omega_o} ( P-P_o) + 
	\left. \dfrac{\partial T_{gen}( P, \omega)}{\partial \omega} \right |_{P_o,\omega_o} (\omega - \omega_o)
\end{equation}
Below the the generator torque sensitivity to the grid power change term from \cref{eq:wtLin_taylor} is derived. From \cref{eq:wtLin_gen_1_1} to \cref{eq:wtLin_gen_1_2} the \textit{sum rule} is used to split the derivative. From \cref{eq:wtLin_gen_1_2} to \cref{eq:wtLin_gen_1_3} the first fractions in the denominators of the partial derivatives of the two terms are treated as a product of two functions thus the \textit{product rule} is used. The assumption is that the grid power is completely disconnected from the generator through the converter. Thus from \cref{eq:wtLin_gen_1_3} to \cref{eq:wtLin_gen_1_4} $ \, \dfrac{\partial \, \omega^{-1}}{\partial P} = 0 $.
\begin{align} 
	\dfrac{\partial T_{gen}( P, \omega)}{\partial P} &= \dfrac{\partial \left (\dfrac{P_{loss} +  P}{\omega}\right )}{\partial P} \label{eq:wtLin_gen_1_1} \\
	& = \dfrac{\partial \left (\dfrac{P_{loss}}{\omega} \right )}{\partial P} + \dfrac{\partial \left ( \dfrac{ P}{\omega} \right )}{\partial P} \label{eq:wtLin_gen_1_2} \\
	& = \dfrac{1}{\omega} \cdot \dfrac{\partial P_{loss}}{\partial P} + \dfrac{\partial \left ( \dfrac{1}{\omega} \right )}{\partial P} P_{loss} + \dfrac{1}{\omega} \cdot \dfrac{\partial P}{\partial P} + \dfrac{\partial \left (\dfrac{1}{\omega} \right )}{\partial P}  P \label{eq:wtLin_gen_1_3} \\
	& = \dfrac{1}{\omega} \cdot \dfrac{\partial P_{loss}}{\partial P} + \dfrac{1}{\omega} \label{eq:wtLin_gen_1_4}
\end{align}


The generator torque sensitivity to rotational velocity change from \cref{eq:wtLin_taylor} is then derived:
\begin{align}
	\dfrac{\partial T_{gen}(P, \omega)}{\partial \omega} & = \dfrac{\partial \left (\dfrac{P_{loss} +  P}{\omega}\right )}{\partial \omega} \\
	& = \dfrac{\partial \left (\dfrac{P_{loss}}{\omega} \right )}{\partial \omega} + \dfrac{\partial \left (\dfrac{P}{\omega} \right )}{\partial \omega} \\
	& = \dfrac{1}{\omega} \cdot \dfrac{\partial P_{loss}}{\partial \omega} + \dfrac{\partial \left (\dfrac{1}{\omega} \right)}{\partial \omega} P_{loss} + \dfrac{1}{\omega} \dfrac{\partial P}{\partial \omega} + \dfrac{\partial \left (\dfrac{1}{\omega} \right )}{\partial \omega} P \\
	& = \dfrac{1}{\omega} \cdot  \dfrac{\partial P_{loss}}{\partial \omega} - \dfrac{1}{\omega^2}(P + P_{loss}) + \dfrac{1}{\omega} \dfrac{\partial P}{\partial \omega} \\
	& = -\dfrac{1}{\omega^2}(P + P_{loss}) + \dfrac{1}{\omega} \cdot \dfrac{\partial P_{loss}}{\partial \omega}
\end{align}
The above derived generator model is referred to the low-speed generator side by replacing $ \omega $ with $ \omega_L $ where $ \omega_L = \left (\dfrac{N_r}{N_g} \right ) \omega $. This yields the final linear generator model evaluated at the operating point $ (P_o, \omega_o) $. Furthermore $P_{loss}$ is still just a function of $ \omega_L $ and $ P $:
\begin{equation}
	\begin{split}
		T_{gen}(P, \omega_L) 	& \approx \left. \left ( \dfrac{1}{\omega_L} \cdot \dfrac{\partial P_{loss}(P, \omega)}{\partial P} + \dfrac{1}{\omega_L} \right ) \right |_{P_o,\omega_{L_o}} (P - P_0) \\ 
		& + \left ( -\dfrac{1}{\omega_L^2}(P + P_{loss}(P, \omega)) + \left. \dfrac{1}{\omega_L} \cdot \dfrac{\partial P_{loss}(P, \omega)}{\partial \omega_L} \right ) \right |_{P_o,\omega_{L_o}} (\omega_L - \omega_{L_o})
	\end{split}
\end{equation}


Presently the generator model in wtLin is implemented assuming a stiff drivetrain such that $ \Omega = \omega_L $ and as such the input to the generator model is $ \Omega $.  \todo[]{Er det muligt at ændre modellen så at den tager højde for et flexibelt drivetrain? Giver det overhovedet mening?}


\subsubsection{Unity model converter} \label{sec:wtLin_conv_unity}
The converter is regarded as a source of electrical power whose output is the power which is input to the generator.

The dynamics of modern converters are way faster than the rotor and tower dynamics and therefore it is simply modelled as an algebraic equation as a direct feed-through:
\begin{equation}\label{eq:wtLin_comp_convdft}
	P_{conv} = P_{ref}
\end{equation}
In other words the converter is treated as a \textit{black box} system which, when given a power reference, delivers a power equal to said power reference instantly.


\subsubsection{Aerodynamic torque} \label{sec:wtLin_aero_torque}
In \cref{sec:theory_aero} the rotor torque was defined for a blade by integrating over the torque component of each blade element based on a combination of the lift and drag forces. When calculating the total stationary torque it is convenient to use the pre-calculated power coefficient values $ C_P $ to determine the torque \cite{Knudsen2013}.

In \cref{eq:power_w_Cp} the extractable power from the free wind was defined. When combining this equation with the definition of power in a mechanical system the torque on the rotor can be expressed:
\begin{equation}\label{eq:wtLin_Mrot_lambda}
	M_{rot}(\Omega, \lambda) = \dfrac{1}{2} \rho A_d v_0^3 \, C_p(\theta, \lambda) \dfrac{1}{\Omega}
\end{equation}
$ C_P $ is the power coefficient and table lookups of it are extracted from VTS.

The TSR is dependent on $ \Omega $ and $ v_0 $ and thus the rotor torque ends up being dependent on $ \theta $, $ \Omega $ and $ v_0 $:
\begin{equation}\label{eq:wtLin_Mrot_wind}
	M_{rot}(\theta, \Omega, v_0) = \dfrac{1}{2} \rho A_d v_0^3 \, C_p(\theta, \Omega, v_0) \dfrac{1}{\Omega}
\end{equation}




The model is then linearised at an operating point $ (\theta_o, \Omega_o, v_{0_o}) $:

\begin{align}
	M_{rot}(\theta, \Omega, v_0) \approx M_{rot}(\theta_o, \Omega_o, v_{0_o}) 
	& + \left. \dfrac{\partial M_{rot}(\theta, \Omega, v_0)}{\partial \theta} \right |_{\theta_o, \Omega_o, v_{0_o}} ( \theta-\theta_o) \\
	& + \left. \dfrac{\partial M_{rot}(\theta, \Omega, v_0)}{\partial \Omega} \right |_{\theta_o, \Omega_o, v_{0_o}} ( \Omega-\Omega_o) \\
	& + \left. \dfrac{\partial M_{rot}(\theta, \Omega, v_0)}{\partial v_0} \right |_{\theta_o, \Omega_o, v_{0_o}} ( v_0 - v_{0_o})
\end{align}


\subsubsection{Aerodynamic thrust} \label{sec:wtLin_aero_thrust}
In \cref{sec:wtLin_aero_torque} the stationary rotor torque was calculated based on the pre-calculated power coefficient table. Likewise the stationary rotor thrust can be calculated from the pre-calculated thrust table $ C_T $. Thus the stationary aerodynamic rotor thrust force can be expressed as such:
\begin{equation} \label{eq:wtLin_aero_thrust}
	F_T = \dfrac{1}{2} A \rho A_d v_0^2 C_T(\theta, \Omega, v_0)
\end{equation}
Just like for $ C_P $, table lookups of $ C_T $ are extracted from VTS. $ C_T $ like $ C_P $ is a mapping from the pitch angle, rotor velocity and free wind speed to a total stationary rotor thrust.


\subsubsection{Rotor wind}
There is an interaction between the tower fore-aft movement and the wind speed which ultimately results in a constantly changing wind speed as seen from the rotor's POV. Thus it is necessary to calculate the \textit{free} wind speed as observed from the rotors point of reference. This is simply done by subtracting the free wind speed $ v_0 $ from the hub translational velocity $ v_y $.
\begin{equation}\label{eq:wtlin_comp_rotorwind}
	v_{0_{rot}} = v_{0} - v_y
\end{equation}
$ v_{0_{rot}} $ is \underline{not} the rotor wind which is the wind speed at the rotor plane but the free wind modified by the turbine velocity.


\subsubsection{Fore-aft tower motion}
While most literature makes use of the DOF notation seen in \cref{fig:fowt_coordinates}, vestas uses another notation. Most importantly \textbf{x} and \textbf{y} are interchanged and in

The fore-aft tower motion is both the x-axis translation (surge) 

% Template:
%\begin{equation}\label{eq:wtlin_comp_}
%	
%\end{equation}

\subsection{Sum-up of full model}

\subsubsection{Inputs, states and outputs}
\textbf{States:}
Converter with time constant (maybe used?): $ P_{conv} $

Flexible drivetrain: $ \Omega $ and $ \omega_L $

Fore-aft tower model: $ \ddot{p}_y $ and $ \dot{p}_y $

Side-side tower model (maybe used?): $ \ddot{p}_x $ and $ \dot{p}_x $

Pitch model with dynamics (maybe used?): $ \theta $

FLC: $ \theta_{ref} $

PLC: $ P_{ref} $

State vector: 
\begin{equation}\label{key}
	x = [\Omega, \omega_L, \ddot{p}_y, \dot{p}_y, \ddot{p}_x, \dot{p}_x, \theta, \theta_{ref}, P_{ref}]^T
\end{equation}
\textbf{Inputs:}


\textbf{Outputs:}